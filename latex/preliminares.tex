\clearpage
\begin{center}
    \thispagestyle{empty}
    \vspace*{\fill}
        Existen pocas metas en la vida profesional y académica de una persona como la investigación científica que implica el doctorado. Más allá de la dedicación personal, el esfuerzo de superación o la curiosidad humana que trata de abarcar un poco más de lo que se puede ver, entender o explicar, existe un cúmulo de personas que sacrifican su tiempo y aportan una cuota mayor de paciencia que el mismo doctorando. Todas estas personas fueron, son y serán el punto de apoyo para todos los empréstitos pasados, este en particular, y todos los proyectos futuros que puedan devenir. Por esa misma razón, estoy dedicando la siguiente tesis doctoral a las personas que la han hecho realidad.
            
        \begin{itemize}
            \item Primero que nada a mi esposa Ángela Alejo, quien ha sido la piedra angular de todo lo que soy y todo lo que seré en el futuro. 
            \item A nuestra hija Ruth Meilij, futura Física Cuántica, y una de las tantas (pocas) personas que entiende de \LaTeX y FORTRAN.
            \item A mis padres Norma Ezeiza y Ricardo Meilij, quienes siempre creyeron que lo lograría y sacrificaron muchos fines de semana para que pudiera estudiar.
            \item A mi jefe Max Harari por prestarme un libro de Big Data en el 2014 y despertar la curiosidad por la Ciencia de Datos. Siempre quise ser un Data Scientist, simplemente no sabía que existía tal cosa hasta que un libro me dejó más preguntas que respuestas. 
            \item A los profesores Roger Peng, Ph.D., Jeff Leek, Ph.D. y Brian Caffo, Ph.D. de \emph{Johns Hopkins University}. Gracias a sus clases y emprendimiento aprendí las habilidades de estadística, programación y matemáticas que me permitieron seguir investigando. 
            \item A los increíbles profesores de la Universidad Benito Juárez G. Es poco probable que entiendan la importancia y evolución que significan para todos los doctorandos del grupo G2V2. Por eso es tan importante que sepan que se está sembrando en estos momentos la nueva generación de investigadores del área de la Administración Gerencial en Latinoamérica gracias a sus esfuerzo. 
        \end{itemize}
    \vspace*{\fill}
    \end{center}
\clearpage
