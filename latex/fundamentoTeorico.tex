\setcounter{chapter}{1}
\chapter{Capítulo II: \\Fundamento Teórico}
\thispagestyle{empty}

\section{Estado del Arte}
A pesar de ser un campo relativamente nuevo, la Ciencia de Datos está profundamente sustentada por la teoría académica (quizás por sus implicaciones como campo multidisciplinario y su importancia para la solución de problemas que impactan otras disciplinas).

De acuerdo a la bibliografía existente, la primera persona en hacer un bosquejo de la idea fue el académico Danés Peter Naur en su libro “Concise Survery of Computer Methods”. Naur sin embargo utiliza el término más que nada para sustituir el de ciencia computacional \cite{naur}. El investigador de Laboratorios Bell y profesor de la Universidad de Princeton, John Tukey, hace un mejor acercamiento al escribir el primer artículo científico sobre como la disciplina de la estadística cambiaba con el advenimiento de la informática \cite{tukey}. Mucho más tarde fue el estadista de la Universidad de Tokio Chikio Hayashi quien definiría de manera sucinta el concepto de Ciencia de Datos como un concepto sintético para unificar la estadística, el análisis de datos y los métodos relacionados con la consecución lo resultados \cite{hayashi}.

Es interesante que los métodos de aprendizaje automatizado proliferaron de forma paralela al concepto de ciencia de datos, y solo fueron absorbidos por esta en los últimos diez años. Alpaydim nos describe el aprendizaje automatizado como la programación de computadoras para optimizar un criterio de desempeño utilizando datos o experiencia pasada \cite{alpaydin}. Tom Mitchell respeta este concepto al describir el aprendizaje automatizado como “… la construcción de programas computacionales que aprenden con la experiencia…” \cite[pag. XV]{mitchell}. Solo Peter Harrington utiliza una descripción mucho más simplista al determinar que “El aprendizaje automatizado es la extracción de información de la data.” \cite[pag. 5]{harrington}.

La teoría detrás de la regresión lineal es bastante homogénea a través de todos los autores. Zumel y Mount describen la regresión lineal como el más común de los métodos de aprendizaje automatizado \cite{zumelMount}, y si no, es muy fácil verificar cual otro método probar como segunda opción. Para Daroczi, el énfasis está en los modelos de regresión multivariable (una extensión de la regresión lineal simple de un solo predictor y resultado) que construyen el camino para la predicción de fenómenos complejos en la naturaleza y negocios \cite{daroczi}. Por su parte, Harrington resume los beneficios de la regresión lineal \cite{harrington} por la facilidad de interpretar los resultados y lo frugal en el uso de ciclos de computación (aunque puede ser menos útil si el fenómeno no es perfectamente lineal).
 
Muchos autores han escrito sobre las series de tiempo, pero es difícil agregar al tema o discutir las ideas del profesor Robert Hyndman, uno de los expertos más respetados en la comunidad de la estadística por su trabajo en las series de tiempo. Hyndman extiende la teoría a las series de tiempo como elementos de pronóstico y su relación con la regresión lineal \cite{hyndman}. Desde el punto de vista técnico, Hyndman es el creador de varias bibliotecas de funciones de pronóstico utilizando series de tiempo y ARIMA en lenguaje R. Dentro de la bibliografía, Daroczi es quien agrega detalles sobre la detección temprana de valores atípicos que pueden dificultar – y mucho – el análisis \cite{daroczi}. Un componente importante de las series de datos es la detección de si son o no auto-regresivas (lo que determina mucho de su poder predictivo). La fórmula para la detección de series auto-regresivas es el test Dickey-Fuller, y la mejor bibliografía es el artículo científico escrito por ambos profesores en la revista especializada Econometrica \cite{dickeyfuller}. A pesar de ser un artículo contemporáneo, la teoría detrás de la prueba Dickey-Fuller nos permite descartar series de tiempo no-regresivas con poco poder de predicción.
 
El uso de modelos ensamblados es en cierta forma la prueba final de la hipótesis de trabajo: la utilización de dos modelos entrecruzados cuyos resultados conforman una tabla temporal de valores esperados de los cuales se genera un nuevo modelo sintético de predicción más general y con mayor capacidad de predicción en juegos de datos de validación cruzada. Este concepto es novel; Witten y Frank lo describen como combinación de métodos múltiples, y escriben: “… un enfoque obvio para hacer mejores decisiones es tomar el resultado de diferentes métodos y combinarlos…” \cite{datamining}. Zhou nos describe que “… los modelos ensamblados que entrenan múltiples variables y luego las combinan para uso de entrenamiento, con el Boosting y el Bagging como representantes principales, representan lo más novedoso en el estado del arte de la ciencia de datos…” \cite[pag. 5]{ensembleMethods}. De una manera un tanto más coloquial, Zhang y Ma describen el uso de modelos ensamblados con una analogía de la vida real, en la cual los pacientes buscan una segunda y hasta tercera opinión de expertos antes de someterse a una operación complicada \cite{ensembleMachineLearning}. Curiosamente tanto Zhang, Ma y Zhou hablan de la combinación de métodos de regresión general con clasificadores, y solo Witten y Frank hablan de otras combinaciones (por supuesto, Witten y Frank comenzaban a escribir en los albores del ensamblaje de métodos, cuando los clasificadores no estaban tan de moda porque el análisis era mayoritariamente de números, algo que cambió con el avance de las redes sociales).

En su libro “Crisis Cambiarias en Países Emergentes” el Dr. Bernardo Carriello utiliza un modelo de descripción (más que de predicción) de corrida de las tasas de cambios, en el cual los regresores incluían variables de medición económicos como crédito privado como porcentaje del PIB, tasa de variación de reservas, desalineación de tipo real, y otros \cite{crisisCambiarias}. Carriello utiliza muchísimo modelos lineales dicotómicos que modelan los escenarios con variables binarias (algo muy común entre los economistas) que por lo general favorecen regresiones logísticas o con la utilización de variables dummy o comodín (se multiplican por el coeficiente uno o cero según tengan o no valor). La mayoría de la bibliografía de aprendizaje automatizado y ciencias de datos prefieren el estudio de variables continuas y reales con amplitud de rango y valores, algo que está más cerca de la disciplina de la bioestadística que de la economía. Una pregunta adicional válida es si tomar metodologías más cercanas a la bioestadística se aplica para la predicción financiera mejor que los modelos dicotómicos actuales.
 
Volviendo a la pregunta mayor de área, el autor y antiguo Ministro de Economía de Colombia, Alfonso Ortega Cárdenas, menciona como material de bibliografía universitaria, la re-valorización del dólar frente al peso colombiano tras el comienzo de la caída de los precios del petróleo a partir del año 2015 \cite{cardenas}. El Dr. Cárdenas no hace mucho hincapié en la correlación de ambas variables, y prefiere ahondar en temas macro-económicos como la variación de la tasa de interés como elemento de presión en la tasa cambiaria y las leyes de ingreso de capital extranjero. Pero es claro que el efecto de las fuentes de ingreso del petróleo como variable clave en el valor final de la TRM ya han sido definidas – si bien algo ligeramente – como claves en un libro de texto de economía de Colombia. ¿Hay elementos adicionales que indiquen la importancia de otras fuentes de ingresos como posibles modeladores y variables de predicción de la TRM? Si los hay, y aparecen en la misma bibliografía de Cárdenas quien describe en detalle a) el sector petrolero, b) el sector siderúrgico, c) el carbón, y d) el níquel.

Los autores Castaño, Callejas, Ochoa y Henao de la Facultad de Ciencias Económicas de la Universidad de Antioquia, hacen un trabajo innovador y de avanzada técnicas estadísticas en su artículo científico \emph{Modelando el Esquema de Intervenciones del Tipo de Cambio para Colombia}. En dicho escrito se determina la eficiencia de las intervenciones realizadas por el Banco de la República empleando el modelo teórico del canal de coordinación. Los autores evalúan el efecto que tiene el diferencial de tasas de interés, la variable de intervenciones construida por medio de un modelo \emph{Markov-switching}, y el procedimiento de inversionistas técnicos y fundamentalistas sobre diferentes cuantiles del retorno de la tasa representativa del mercado (TRM). La metodología utilizada son las regresiones de cuantiles bajo redes neuronales \cite{modelandoIntervenciones}. Castaño, Callejas, Ochoa y Henao no son los únicos en utilizar redes neuronales, o una variante de redes neuronales, para entrenar modelos de predicción. Los profesores de la Universidad de Parana, Brasil, Scarpin y Steiner utilizan un modelo de Redes Neuronales Radiales Artificiales para un modelo de pronóstico de reemplazo de artículos vendidos en supermercados \cite{scarpinSteiner}. Sin embargo el modelo propuesto por Scarpin y Steiner necesita alimentarse no solo de la data actualizada de movimientos de producto, sino de un pronóstico inicial de necesidades de niveles de venta. Con una base similar - la necesidad de un pronóstico a priori - pero una metodología diferente, los doctores Wang y Xu crean un marco para pronosticos cooperativos utilizando modelos basados en Pronósticos Combinados de Bayes \cite{wangXu}. 
 
Los investigadores Mehreen Rehman, Gul Muhammad Khan y Sahibzada Ali Mahmud han utilizado la ciencia de datos para la predicción de FOREX. Los autores utilizan CGP (Programación Genética Cartesiana), una extensión del uso de redes neuronales, para obtener predicciones del dólar australiano con 98.72\% de precisión por períodos extendidos de hasta 1,000 días \cite{rehmanKhanMahmud}. Los autores alimentan el sistema CGP con información histórica de las monedas en cuestión compuesta por 500 días de cotización.
 
Quizás menos conocido es el uso de clasificadores versus regresores. Este camino toma el estudio de los doctores Hossein Talebi, Winsor Hoang y Marina Gavrilova. En su investigación en búsqueda de la mejora de sistemas automatizados de corretaje de FOREX utilizando aprendizaje automatizado, los autores proponen un nuevo método de clasificación. Dicho método utiliza extracción de clasificadores de múltiples escalas para el entrenamiento de datos, y luego se ensamblan diferentes clasificadores por voto Bayes \cite{talebiHoang}. El método propuesto demuestra superioridad a la hora de ensamblar clasificadores por encima de clasificadores individuales.
 
Otro estudio interesante es el de los profesores de matemática de la universidad de Beijing Lean Yu, Shouyang Wang, y K. K. Lai. El enfoque es novedoso en el sentido que utilizan un sistema ensamblado de auto-regresión lineal generalizada (GLAR) con redes neuronales artificiales (ANN). Los autores llegan a la conclusión que los resultados en las predicciones son superiores a los resultados de las predicciones de los métodos por separado, o de métodos similares con regresiones lineales \cite{yuWangLai}. Una lectura cuidadosa de los resultados evidencia márgenes de error del 1.56\% al 3.57\%, dependiendo de la moneda a evaluar.

\section{Marco Teórico}
La Ciencia de Datos se caracteriza por ser una ciencia multidisciplinaria. A la par de extenso conocimiento de estadística y programación, el científico de datos debe poseer un dominio extremo sobre el campo de acción o \textit{domain expertise} como se le conoce en inglés \cite{pengMatsui}. Este portafolio de conocimiento se refleja de igual manera en el marco teórico del trabajo de investigación, que debe unir, analizar y sintetizar la teoría de la estadística descriptiva e inferencial, el aprendizaje automatizado y los modelos ensamblados para la resolución del problema, y la economía de Colombia para entender el problema en toda su magnitud. Por tales razones se ha decidido desglosar el marco teórico en seis secciones diferentes, cada una con su base de conocimiento distintivo. 

\begin{enumerate}
    \item La Economía de Colombia
    \item Regresión Lineal
    \item Series de Tiempo
    \item La Ciencia de Datos
    \item Aprendizaje Automatizado
    \item Modelos Ensamblados
\end{enumerate}

Cada una de estas secciones se unen en un todo final para una solución holística del problema de investigación.
