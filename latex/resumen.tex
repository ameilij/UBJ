\pagenumbering{gobble}

\Large
 \begin{center}
Resumen\\ 
\hspace{5pt}

% Author names and affiliations
\large
Ariel E. Meilij$^1,^2$  \\
\hspace{5pt}

\small  
$^1$) Universidad Benito Juárez G.
$^2$) Universidad Latinoamericana de Ciencia y Tecnología
\end{center}
\hspace{5pt}

\normalsize
El siguiente trabajo de investigación tiene como finalidad determinar un modelo de predicción para la tasa de cambio del dólar en Colombia (conocida legalmente como la TRM o Tasa Representativa de Mercado) utilizando aprendizaje automatizado. La investigación se enfoca en la hipótesis de que la serie de tiempo que representa la TRM puede utilizarse como elemento de predicción por contener la tendencia de una variable macroeconómica, pero el modelo es más robusto cuando se combina con las variables exógenas que intervienen y coaccionan dicho comportamiento. Para la TRM estas variables independientes son los regresores representados por los principales rubros de exportación del país. A través del aprendizaje automatizado (machine learning) se entrenan los datos que representan las distintas series de tiempo para generar un modelo de pronóstico ARIMA en función de la fluctuación inherente de la TRM, y un modelo de regresión multivariable utilizando la TRM como variable dependiente y los diferentes datos de rubro de exportación como variables independientes. Ambos modelos entrenados representan pronósticos de alta precisión cuyos resultados se convierten en variables independientes de un modelo ensamblado que utiliza las salidas de los primeros como entradas para la generación de un nuevo aprendiz con valores superiores de precisión. 

\textbf{Palabras Claves:} TRM, tasa de cambio, aprendizaje automatizado, ARIMA, regresión multivariable, modelos ensamblados, ciencia de datos


\hspace{10pt}

\Large
 \begin{center}
Abstract\\ 
\hspace{5pt}
\end{center}

\normalsize
The objective of the following research thesis is the determination of a prediction model for the Colombian Peso exchange rate (legally known as the TRM or Market Representative Rate) using machine learning. The investigation focuses on the hypothesis that the time series representative of the TRM can be used as a prediction element for containing macroeconomic tendencies of the variable, yet the model becomes more robust when combined with the exogenous variables that intervene and enforce such behavior. For the TRM said independent variables are regressors representatives of the main export commodities for the country. Through machine learning data representing the different time series are used to generate an ARIMA forecasting model in function of the inherent fluctuation of the TRM, and a multivariable regression model using the TRM as the dependent variable and the different export commodities as independent regressors. Both trained models represent highly accurate forecasting tools whose results become the independent variables for a third ensemble model through stacking techniques, where the output of the first two learners become the input of a third with a higher level of accuracy.

\textbf{Key Words:} TRM, forex, machine learning, ARIMA, multivariable regression, ensambled models, data science