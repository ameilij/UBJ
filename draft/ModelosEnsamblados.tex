\section{Modelos Ensamblados}
El tema de modelos ensamblados es uno que por lo general se reserva más como técnica de composición que cómo teoría del aprendizaje automatizado. 

\subsection{Introducción}
El uso de modelos ensamblados es en cierta forma la prueba final de la hipótesis de trabajo: la utilización de dos modelos entrecruzados cuyos resultados conforman una tabla temporal de valores esperados de los cuales se genera un nuevo modelo sintético de predicción más general y con mayor capacidad de predicción en juegos de datos de validación cruzada. Este concepto es novel; Witten y Frank lo describen como combinación de métodos múltiples, y escriben: “… un enfoque obvio para hacer mejores decisiones es tomar el resultado de diferentes métodos y combinarlos\ldots” (Witten, I. y Frank, E., 2005). Zhou nos describe que “… los modelos ensamblados que entrenan múltiples variables y luego las combinan para uso de entrenamiento, con el Boosting y el Bagging como representantes principales, representan lo más novedoso en el estado del arte de la ciencia de datos…” (Zhou, Z., 2012, pg. VII). De una manera un tanto más coloquial, Zhang y Ma describen el uso de modelos ensamblados con una analogía de la vida real, en la cual los pacientes buscan una segunda y hasta tercera opinión de expertos antes de someterse a una operación complicada (Zhang, C. Y Ma, Y., 2012). Curiosamente tanto Zhang, Ma y Zhou hablan de la combinación de métodos de regresión general con clasificadores, y solo Witten y Frank hablan de otras combinaciones (por supuesto, Witten y Frank comenzaban a escribir en los albores del ensamblaje de métodos, cuando los clasificadores no estaban tan de moda porque el análisis era mayoritariamente de números, algo que cambió con el avance de las redes sociales). 

\subsection{Combinando Métodos}
La combinación de métodos es el ultimo paso en la estrategia de construcción de un sistema ensamblado de aprendizaje automatizado. La pregunta de que métodos combinar esta estrechamente relacionado con el tipo de juegos de datos y la solución que se busca alcanzar. Por ejemplo, alguno métodos de clasificación como los vectores de soporte solo devuelven valores discretos \cite{ensembleMachineLearning}. De tal manera el uso de dos métodos alternos en uno ensamblado estará determinado por la forma final en que se ensamblan y el algoritmo final utilizado para la decisión de predicción. Tanto Polikar \cite{ensembleMachineLearning} como Zhou \cite{ensembleMethods} citan como preferibles las metodologías de voto por mayoría, promedio, promedio ponderado, y ensamblaje infinito. 

\subsection{Diversidad}
La diversidad de ensamblaje, o la diferencia entre diferentes métodos de aprendizaje, es un tema fundamental en el ensamblaje de métodos \cite{ensembleMethods}. Intuitivamente es fácil entender que para obtener una ventaja de la combinación, es necesario que los aprendizajes sean diferentes, de otra manera la ganancia en desempeño no seria marginalmente superior a los métodos por separado \cite{ensembleMethods}.

\subsection{Bagging}
La idea del \emph{bagging} esta estrechamente ligada al \emph{bootstrapping}, y determinada por la selección de múltiples muestras de datos generadas a través de \emph{bootstrapping}, utilizadas para alimentar clasificadores, sobre cuyos resultados el método ensamblado puede votar \cite{daume}.]

\subsection{Boosting}
El \emph{boosting} es la técnica por la cual se toma un algoritmo de aprendizaje con malos resultados (técnicamente conocido como un clasificador débil) y se lo transforma en un clasificador fuerte. La forma en la cual funciona el \emph{boosting} es que basado en un juego de datos y resultados pasados, va generando nuevas predicciones. Las predicciones con resultados aceptables se les pone menor peso y recursos, mientras que el algoritmo vuelve a iterar en aquellas predicciones con valores lejanos hasta que cobran fuerza \cite{daume}. Esta técnica recibe el nombre de \textbf{AdaBoost}, del ingles \emph{adaptive boosting algotithm}. Esta fue una de las primeras técnicas practicas en la ciencia de datos.
