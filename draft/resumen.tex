\pagenumbering{gobble}

\Large
 \begin{center}
Resumen\\ 
\hspace{5pt}

% Author names and affiliations
\large
Ariel E. Meilij$^1,^2$  \\
\hspace{5pt}

\small  
$^1$) Universidad Benito Juárez G.\\
ariel.meilij@gmail.com\\
$^2$) Universidad Latinoamericana de Ciencia y Tecnología
\end{center}
\hspace{5pt}

\normalsize
El siguiente trabajo de investigación tiene como finalidad determinar un modelo de predicción para la tasa de cambio del dólar en Colombia (conocida legalmente como la TRM) utilizando aprendizaje automatizado. La investigación se enfoca en la hipótesis de que las series de tiempo pueden utilizarse como elementos de predicción por contener la tendencia de una variable, pero son más fuertes cuando se utilizan las variables exógenas que modelan e intervienen en dicho comportamiento. Para la TRM estas variables independientes son los regresores representados por los principales rubros de exportación del país. En vez de extraer un modelo en base a la observación, se utiliza el aprendizaje automatizado para entrenar los datos de las series de tiempo de la TRM y los rubros de exportación, y se unen con un modelo ensamblado de pronostico ARIMA para la serie de tiempo y regresión multivariable para los factores exógenos. 

\hspace{10pt}

\Large
 \begin{center}
Abstract\\ 
\hspace{5pt}
\end{center}

\normalsize
The following research thesis objective is the determination of a prediction model for the Colombian Peso exchange rate (legally known as the TRM) using machine learning techniques. The research focuses on the hipothesis that time series can be used for forecasting purposes given their intrinsic trend capacity, but their potential can be strengthened when coupled with exogenous variables that confound said behavior. For the TRM these independent variables are represented by the principal export components. Instead of extracting a model through observation, machine learning is used to train the different time series representing the values of the TRM and exports. Ensambled methodologies of stacking are then used to join the predictive capacities of ARIMA forecasting for TRM time series and multivariable regresion for the different export variables confounding the TRM pricing.