\subsection{Marco Conceptual}
La TRM representa un ejemplo perfecto de series de tiempo con marcada tendencia secular y estacionalidad. Pudiéramos en cualquier caso utilizar métodos de pronóstico de series de tiempo como ARIMA para entonces estimar el valor futuro de la TRM. Sin embargo, esta estimación tendría como único elemento de referencia el valor de la TRM en diferentes puntos del tiempo. Estaríamos resolviendo el problema haciendo caso omiso de las diferentes variables exógenas que intervienen en la economía mundial y de Colombia, y que juntas definen a través de la ley de oferta y demanda el valor final de la TRM.
 
En dicho caso, el uso de regresión lineal y regresión multivariable nos permite justamente crear un modelo de pronóstico basado en regresores relacionados con la variable independiente que a su vez son variables exógenas. A través de la creación de dicho modelo la predicción se hace sin tomar en cuenta que la TRM es una serie de tiempo y puede aportar en su descomposición factores importantes tales como la estacionalidad de la tendencia.
 
Si partimos del hecho de que la TRM es una serie de tiempos y puede ser estudiada como tal, y que además los diferentes rubros de exportación de la economía de Colombia aportan divisas que a través de la ley de oferta y demanda regulan en el mercado sus precios, es una admisión de que existen variables relacionadas, aunque exógenas que rigen el comportamiento de la TRM además de sus tendencias seculares y estacionalidad.
 
Es válido, si partimos de ambos supuestos y los tomamos como ciertos, pensar que puede existir un modelo híbrido de pronóstico que tome en consideración ambos eventos. Hay claras sinergias entre las exportaciones de Colombia, fuentes de ingresos de divisas, que se manifiestan en indicios de comportamiento de cotización de mercado. Hay claras manifestaciones en la estacionalidad y tendencia de una serie de tiempos representativa de la TRM. Ambas metodologías de pronóstico son el reflejo de la realidad económica modelados de diferentes maneras.
 
La Ciencia de Datos utiliza el aprendizaje automatizado como forma de llegar a modelos complejos de clasificación y estimación. Es aprendizaje automatizado porque los datos son los que se entrenan y definen el modelo, no el científico. Si existen sinergias fuertes el modelo cobra vida de forma automática. Inclusive el aprendizaje automatizado prevé la existencia de modelos ensamblados, justamente para aumentar la capacidad de estimación cuando un fenómeno puede ser modelado mejor como una sumatoria de clasificadores y/o regresores que por un solo método. Por lo tanto, es plausible obtener un modelo de predicción de la TRM generado por el aprendizaje automatizado de series de tiempo y regresión multivariable en un solo método híbrido y ensamblado que parta desde:
 
\begin{itemize}
\item el uso de los valores de la TRM como serie de tiempo,
\item y los valores de la TRM y los diferentes rubros mayores que componen la canasta de exportación de Colombia como una estructura de datos para el análisis de regresión multivariable.
\end{itemize}

¿Cómo podemos predecir la TRM para mitigar el efecto negativo de las fluctuaciones en la tasa de cambio en la contabilidad de precios y costos?

La hipótesis de trabajo para resolver dicha pregunta postula un modelo basado en los principales rubros de exportación de Colombia diseñado con Machine Learning de los mismos datos:
\begin{itemize}
\item Existen una cantidad finita - e inferior a la decena - de productos de exportación que fungen como variables de agregación al producto bruto interno de Colombia y que son necesarias para la consecución de un modelo predictivo parsimonioso de la TRM.
\item El valor de la TRM, tal cual lo ja la Superintendencia Financiera de Colombia, no es sino el reflejo de los movimientos de estas variables de aportación que ayudan a modelar y controlar la tasa de cambio.
\item El comportamiento pasado de dichas variables puede ser utilizado para entrenar y generar un modelo estadístico predictivo parsimonioso utilizando aprendizaje automatizado cuyo margen de error sea inferior al 5\% (o, en otros términos, p < 0,05).
\item El modelo final no es único sino es el resultado del ensamblaje de varios modelos matemáticos predictivos y dinámico en su concepción ya que puede ser afectado por la acumulación de nuevos datos de retroalimentación a posteriori 
\end{itemize}

En términos formales:

\[ p^{c}(x) = (p^{c}(c_{1} \mid x),p^{c}(c_{2} \mid x)) \]

Donde:

\[ c_{1} : f(y) = \beta_{0} + \beta_{1}x_{1} + \beta_{2}x_{2} + \cdots + \beta_{n}x_{n} + \epsilon \]
\[ c_{2} : f(y_{t}) = \beta_{0} + \beta_{1}x_{t-1} + \beta_{2}x_{t-2} + \cdots + \beta_{n}x_{t-n} + \epsilon_{t}\]

En términos coloquiales, el modelo de predicción está determinado por el ensamblaje acumulado de dos clasificadores (ambos regresores), un modelo de regresión lineal y otro de pronóstico ARIMA.

La hipótesis de trabajo propuesta tiene marcada diferencias con los trabajos de otros investigadores en el área de predicción del FOREX. 

\begin{itemize}
\item Los investigadores Mehreen Rehman, Gul Muhammad Khan y Sahibzada Ali Mahmud han utilizado la ciencia de datos para la predicción de FOREX. Los autores utilizan CGP (Programación Genética Cartesiana), una extensión del uso de redes neuronales, para obtener predicciones del dólar australiano (Rehman, M., Khan, G. y Mahmud, S., 2014). Este modelo se alimenta exclusivamente de datos históricos de la cotizaciones de la moneda y no tiene datos de variables exógenas como los regresores propuestos de los rubros de exportación de Colombia. 
\item El estudio de los doctores Hossein Talebi, Winsor Hoang y Marina Gavrilova busca la mejora de sistemas automatizados de corretaje de FOREX. Los autores proponen un nuevo método de clasificación con extracción de clasificadores de múltiples escalas para el entrenamiento de datos, y luego se ensamblan diferentes clasificadores por voto Bayes (Talebi, H., Hoang, y Gavrilova, M., 2014). El trabajo tiene en cuenta el uso de métodos ensamblados pero utiliza como clasificadores pares de cotizaciones de monedas (por ejemplo el par COP:USD) las cuales pueden pensarse como un ensamble de dos series de tiempo sin uso de variables exógenas. 
\item Los profesores de matemática de la universidad de Beijing Lean Yu, Shouyang Wang, y K. K. Lai usan un enfoque novedoso en el sentido que utilizan un sistema ensamblado de auto-regresión lineal generalizada (GLAR) con redes neuronales artificiales (ANN). Este estudio es similar a la propuesta de esta investigación pero sigue alimentando el sistema solo con series de tiempo.  
\end{itemize}

Los autores que ya han utilizado aprendizaje automatizado y métodos ensamblados todos recurren al uso de series de tiempo como entradas, sin que ninguno se haya preguntado si existen sinergias adicionales en el concepto de valorización del FOREX. El problema ha sido estudiado con mucho detenimiento desde el punto de vista del aprendizaje automatizado puro pero no desde el punto de vista macro económico y comercial. La hipótesis de trabajo se basa en la observación de que son las exportaciones las que regulan el precio de la TRM y pueden aportar un elemento de agregado de precisión al modelo predictivo si se combina con el modelo entrenado de series de datos. El papel del aprendizaje automatizado es justamente este: entrenar un modelo que minimiza el error de predicción en base a una gran cantidad de datos y facilitar el análisis estadístico del modelo predictivo en aras de lograr un modelo de producción que pueda ser utilizado con facilidad todos los días, varias veces al día de ser necesario, o inclusive miles de veces al día en un sistema automatizado de FOREX. 
